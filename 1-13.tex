\documentclass[a4paper,11pt]{jsarticle}

%%%%%%%%%%%%%%%%%%%%%%%%%%%%%%%%恒例
\usepackage{multicol}
\usepackage{float}
\usepackage{graphicx}
\usepackage{url}
\usepackage{amsmath,cases,amssymb}
\usepackage{braket}
\allowdisplaybreaks %次ページに渡る数式許可
%%%%%%%%%%%%%%%%%%%%%%%%%%%%%%%%

%%%%%%%%%%%%%%%%%%%%%%%%%%%%%%%%定理環境
\usepackage{amsthm}

\newtheorem{Def}{定義}
\newtheorem{Th}{定理}
\newtheorem{Cor}[Th]{系}
\newtheorem{Lem}[Th]{補題}
\newtheorem{Prop}{命題}

\newtheorem{eg}{例}[section]
\newtheorem{prob}{問題}
\newtheorem{property}{性質}

\newsavebox{\probbox}
\newcommand{\EndIt}{}

\newenvironment{barred}[2][]{%
  \renewcommand{\EndIt}{\end{#2}}%
  \begin{#2}#1\noindent\newline
  \begin{lrbox}{\probbox}%
  \addtolength{\linewidth}{-12pt}%
  \begin{minipage}[t]{\linewidth}%
  \nointerlineskip
}{%
  \par
  \xdef\ResetDepth{\prevdepth=\the\prevdepth\relax}%
  \unskip\unskip\unpenalty
  \end{minipage}%
  \end{lrbox}%
  \noindent
  \makebox[\columnwidth]{%
    \hfill\vrule\hspace{2pt}\vrule\hfill
    \usebox{\probbox}}%
  \EndIt
  \ResetDepth
}
%%%%%%%%%%%%%%%%%%%%%%%%%%%%%%%%


%%%%%%%%%%%%%%%%%%%%%%%%%%%%%%%%A4を広く使う
\setlength{\topmargin}{-20pt}
\setlength{\textheight}{20cm}
%%%%%%%%%%%%%%%%%%%%%%%%%%%%%%%%


%%%%%%%%%%%%%%%%%%%%%%%%%%%%%%%%図表式番号表示の書き換え
\makeatletter
 \renewcommand{\theequation}{%
   \arabic{equation}}
  \@addtoreset{equation}{section}
 \makeatother

\makeatletter
 \renewcommand{\thefigure}{%
   \arabic{figure}}
  \@addtoreset{figure}{section}
 \makeatother


\makeatletter
 \renewcommand{\thetable}{%
   \arabic{table}}
  \@addtoreset{table}{section}
 \makeatother
%%%%%%%%%%%%%%%%%%%%%%%%%%%%%%%%

%%%%%%%%%%%%%%%%%%%%%%%%%%%%%%%%ページ番号
\usepackage{fancyhdr}
\usepackage{lastpage}

\pagestyle{fancy}

\lhead{}
\chead{}
\rhead{}

\cfoot{}
\rfoot{\thepage{}/{}\pageref{LastPage}}

\renewcommand{\headrulewidth}{0pt}

%%%%%%%%%%%%%%%%%%%%%%%%%%%%%%%%

%%%%%%%%%%%%%%%%%%%%%%%%%%%%%%%%送り仮名(ルビ)\ruby{漢字}{かんじ}
\newcommand{\ruby}[2]{%
\leavevmode
\setbox0=\hbox{#1}%
\setbox1=\hbox{\tiny #2}%
\ifdim\wd0>\wd1 \dimen0=\wd0 \else \dimen0=\wd1 \fi
\hbox{%
\kanjiskip=0pt plus 2fil
\xkanjiskip=0pt plus 2fil
\vbox{%
\hbox to \dimen0{%
\tiny \hfil#2\hfil}%
\nointerlineskip
\hbox to \dimen0{\mathstrut\hfil#1\hfil}}}}
%%%%%%%%%%%%%%%%%%%%%%%%%%%%%%%%

%%%%%%%%%%%%%%%%%%%%%%%%%%%%%%%%自己定義
\newcommand{\C}[1]{\quad\cdots\text{#1}}	%式に説明文&\C{explain}
\newcommand{\tr}{\text{tr}}	%トレース\tr
\newcommand{\identity}{\mbox{1}\hspace{-0.25em}\mbox{l}}	%恒等演算子\identity
\newcommand{\Slash}[1]{{\ooalign{\hfil$#1$\hfil\crcr\raise.167ex\hbox{/}}}}
%%%%%%%%%%%%%%%%%%%%%%%%%%%%%%%%

%%%%%%%%%%%%%%%%%%%%%%%%%%%%%%%%最終行にのみ式番号を振る alignL 環境
\makeatletter
\def\alignL{\csname align*\endcsname}
\def\endalignL{\refstepcounter{equation}\tag{\number\c@equation}%
  \csname endalign*\endcsname}
\makeatother
%%%%%%%%%%%%%%%%%%%%%%%%%%%%%%%%

\begin{document}
\begin{barred}[フィボナッチ数の一般項]{Prop}
フィボナッチ数の一般項は$\phi = \frac{1 + \sqrt{5}}{2}, \psi = \frac{1 - \sqrt{5}}{2}$として
\begin{align}
F_n = \frac{\phi^n - \psi^n}{\sqrt{5}}
\end{align}
と書ける。
\end{barred}
\begin{proof}
略。ググったら簡単に出てくる。
\end{proof}

\begin{barred}[フィボナッチ数]{Prop}
$F_n$は$\frac{\phi^n}{\sqrt{5}}$に最も近い整数
\end{barred}

\begin{proof}
\begin{align}
&F_nは\frac{\phi^n}{\sqrt{5}}に最も近い整数\\
\Leftrightarrow& F_n = \left[ \frac{\phi^n}{\sqrt{5}} + \frac{1}{2} \right] ( [] はガウス記号 ) \\
\Leftrightarrow& F_n は \left[ \frac{\phi^n}{\sqrt{5}} + \frac{1}{2} \right] -1 < F_n < \left[ \frac{\phi^n}{\sqrt{5}} + \frac{1}{2} \right] + 1 を満たす整数
\label{proof}
\end{align}

(i) $n = 2\bold{\mathcal{N}}$のとき

$0<\frac{\psi^n}{\sqrt{5}}<1/2$より
\begin{align}
\left[ \frac{\phi^n}{\sqrt{5}} + \frac{1}{2} \right] -1
 &= \left[ \frac{\phi^n}{\sqrt{5}} - \frac{1}{2} \right]
 \le \frac{\phi^n}{\sqrt{5}} - \frac{1}{2} \\
 &< \frac{\phi^n}{\sqrt{5}} - \frac{\psi^n}{\sqrt{5}} = F_n\\
 &< \frac{\phi^n}{\sqrt{5}}
 \le \left[ \frac{\phi^n}{\sqrt{5}} + \frac{1}{2} \right] + 1
\end{align}
より式\eqref{proof}が成立。

(ii) $n = 2\bold{\mathcal{N}} - 1$のとき

$-1/2<\frac{\psi^n}{\sqrt{5}}<0$より
\begin{align}
\left[ \frac{\phi^n}{\sqrt{5}} + \frac{1}{2} \right] -1
 &\le \frac{\phi^n}{\sqrt{5}} \\
 &< \frac{\phi^n}{\sqrt{5}} - \frac{\psi^n}{\sqrt{5}} = F_n\\
 &< \frac{\phi^n}{\sqrt{5}} + \frac{1}{2}
 \le \left[ \frac{\phi^n}{\sqrt{5}} + \frac{1}{2} \right] + 1
\end{align}
より式\eqref{proof}が成立。

(i), (ii)より$\forall n \in \mathcal{N}$に対し式\eqref{proof}が成立する i.e. $F_n$は$\frac{\phi^n}{\sqrt{5}}$に最も近い整数
\end{proof}

\begin{proof}
別証。
\begin{align}
&F_nは\frac{\phi^n}{\sqrt{5}}に最も近い整数\\
\Leftrightarrow& \left| F_n - \frac{\phi^n}{\sqrt{5}} \right| < \frac{1}{2}
\label{another_proof}
\end{align}
より式\eqref{another_proof}を示せばよい。
\begin{align}
\left| F_n - \frac{\phi^n}{\sqrt{5}} \right|
= \left| \frac{\psi^n}{\sqrt{5}} \right|
< \frac{\psi^0}{\sqrt{5}}
< \frac{1}{2}
\end{align}

\end{proof}

\end{document}